\documentclass[journal,12pt,twocolumn]{IEEEtran}
%

\usepackage{setspace}
\usepackage{gensymb}
\singlespacing

\usepackage{amsmath}
\usepackage{amsthm}
\usepackage{txfonts}
\usepackage{cite}
\usepackage{enumitem}
\usepackage{mathtools}
\usepackage{hyperref}
\usepackage{listings}
    \usepackage{color}                                            %%
    \usepackage{array}                                            %%
    \usepackage{longtable}                                        %%
    \usepackage{calc}                                             %%
    \usepackage{multirow}                                         %%
    \usepackage{hhline}                                           %%
    \usepackage{ifthen}                                           %%
  %optionally (for landscape tables embedded in another document): %%
    \usepackage{lscape}     
\usepackage{multicol}
\usepackage{chngcntr}
\renewcommand\thesection{\arabic{section}}
\renewcommand\thesubsection{\thesection.\arabic{subsection}}
\renewcommand\thesubsubsection{\thesubsection.\arabic{subsubsection}}

% correct bad hyphenation here
\hyphenation{op-tical net-works semi-conduc-tor}
\def\inputGnumericTable{}                                 %%

\lstset{
%language=C,
frame=single, 
breaklines=true,
columns=fullflexible
}

\begin{document}
%


\newtheorem{theorem}{Theorem}[section]
\newtheorem{problem}{Problem}
\newtheorem{proposition}{Proposition}[section]
\newtheorem{lemma}{Lemma}[section]
\newtheorem{corollary}[theorem]{Corollary}
\newtheorem{example}{Example}[section]
\newtheorem{definition}[problem]{Definition}
\newcommand{\BEQA}{\begin{eqnarray}}
\newcommand{\EEQA}{\end{eqnarray}}
\newcommand{\define}{\stackrel{\triangle}{=}}
\bibliographystyle{IEEEtran}
\providecommand{\mbf}{\mathbf}
\providecommand{\pr}[1]{\ensuremath{\Pr\left(#1\right)}}
\providecommand{\qfunc}[1]{\ensuremath{Q\left(#1\right)}}
\providecommand{\sbrak}[1]{\ensuremath{{}\left[#1\right]}}
\providecommand{\lsbrak}[1]{\ensuremath{{}\left[#1\right.}}
\providecommand{\rsbrak}[1]{\ensuremath{{}\left.#1\right]}}
\providecommand{\brak}[1]{\ensuremath{\left(#1\right)}}
\providecommand{\lbrak}[1]{\ensuremath{\left(#1\right.}}
\providecommand{\rbrak}[1]{\ensuremath{\left.#1\right)}}
\providecommand{\cbrak}[1]{\ensuremath{\left\{#1\right\}}}
\providecommand{\lcbrak}[1]{\ensuremath{\left\{#1\right.}}
\providecommand{\rcbrak}[1]{\ensuremath{\left.#1\right\}}}
\theoremstyle{remark}
\newtheorem{rem}{Remark}
\newcommand{\sgn}{\mathop{\mathrm{sgn}}}
\providecommand{\abs}[1]{\lvert#1\rvert}
\providecommand{\res}[1]{\Res\displaylimits_{#1}} 
\providecommand{\norm}[1]{\lVert#1\rVert}
\providecommand{\mtx}[1]{\mathbf{#1}}
% \providecommand{\mean}[1]{E\left[ #1 \right]}
\providecommand{\fourier}{\overset{\mathcal{F}}{ \rightleftharpoons}}
\providecommand{\system}{\overset{\mathcal{H}}{ \longleftrightarrow}}
\newcommand{\solution}{\noindent \textbf{Solution: }}
\newcommand{\cosec}{\,\text{cosec}\,}
\providecommand{\dec}[2]{\ensuremath{\overset{#1}{\underset{#2}{\gtrless}}}}
\newcommand{\myvec}[1]{\ensuremath{\begin{pmatrix}#1\end{pmatrix}}}
\newcommand{\cmyvec}[1]{\ensuremath{\begin{pmatrix*}[c]#1\end{pmatrix*}}}
\newcommand{\mydet}[1]{\ensuremath{\begin{vmatrix}#1\end{vmatrix}}}
\newcommand{\proj}[2]{\textbf{proj}_{\vec{#1}}\vec{#2}}
\newcommand{\RNum}[1]{\uppercase\expandafter{\romannumeral #1\relax}}
\let\StandardTheFigure\thefigure
\let\vec\mathbf
\title{
\LARGE SM5083\\
    \LARGE Assignment 2 \\[0.5em]
    
    \large Ravi Kumar\par
    \large   SM21MTECH12010  \par
}
\maketitle
\renewcommand{\thefigure}{\theenumi}
\renewcommand{\thetable}{\theenumi}
\section{ chapter \RNum{3}  Miscellaneous examples \RNum{6} Q9}
\item \\If the lines 
$y=x \tan (\frac{11\pi}{24})$,  $y=x \tan (\frac{19\pi}{24})$ \\be at right angles, Show that the angle between the axes is $(\frac{\pi}{4})$ 

\solution
\begin{align}
    ~\text{Given equation of two lines are}~\\ 
    \label{Equation 1}
    \vec{y} = x \tan (\frac{11\pi}{24})\\
    \label{Equation 2}
    \vec{y} = x \tan (\frac{19\pi}{24})
\end{align}
Let the equations of the straight lines $AB$ and $CD$ are i.e.
\begin{align}
    \vec{y} = m_1 ( \vec{x} + \vec{c_1} )\ ~\text{and}~
    \vec{y} = m_2 ( \vec{x} + \vec{c_2} )\
{respectively,}
\end{align}
intersect at a point P and make angles $\theta$1 and $\theta$2 respectively with the positive direction of x-axis\\
So the angle $\theta$ between the lines having slope $m_1$ and $m_2$ is given by
\begin{align}
\label{Equation 3}
\vec{\tan\theta} = \frac{(m_2 - m_1)}{1 + m_1m_2}
\end{align}
Clearly, the slope of the line AB and CD are $m_1$ and $m_2$ respectively.\\
From equations  
    \eqref{Equation 1} and \eqref{Equation 2}
\begin{align}
    \vec{m_1} =  \tan (\frac{11\pi}{24})\\
    \vec{m_2} =  \tan (\frac{19\pi}{24})
\end{align}
Now, put the value of $m_1$ and $m_2$ in equations  
    \eqref{Equation 3}
\begin{align}
 \vec{\tan\theta} = \frac{\tan (\frac{19\pi}{24}) -\tan (\frac{11\pi}{24})}{1 + \tan (\frac{19\pi}{24})\tan (\frac{11\pi}{24})}
\end{align}
Using Formula 
\begin{align}
 \vec{\tan(A+B)} = \frac{\tan (A) -\tan (B)}{1 + \tan (A)\tan (B)}
\end{align}
Then,
\begin{align}
 \vec{\tan\theta} = {\tan (\frac{19\pi}{24}+\frac{11\pi}{24})}\\
 \vec{\tan\theta} = {\tan (\frac{30\pi}{24})}\\
 \vec{\tan\theta} = {\tan (\frac{5\pi}{4})}\\
 \vec{\tan\theta} = {\tan ({\pi}+\frac{\pi}{4})}
 \end{align}
 So,
 \begin{align}
 \vec{\tan\theta} = {\tan (\frac{\pi}{4})}
\end{align}
Hence, the angle between the axes is $(\frac{\pi}{4})$

\end{document}